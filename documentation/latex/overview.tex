% !TeX spellcheck = en_GB
%-------------------------------------------------------------------------------
% File: overview.tex
%       Vehicle ADVISE project documentation.
%
% Author: Yuri Mazzuoli, Francesco Iemma, Marco Pinna
%         Created on 30/06/2021
%-------------------------------------------------------------------------------
\chapter{Overview}\label{ch:overview}

In this chapter a general overview of the scenarios and the involved actors is given.\\
The final goal for each one of the attackers it to gain access to the back-end servers related to self-driving vehicles. This goal can be reached by means of different knowledges, skills and types of attacks, which all differ from one adversary to the other.

\vspace{0.2cm}
\noindent The attack steps and the goals analyzed in this report are based on \textit{"Agreement Concerning the Adoption of Harmonized Technical United Nations Regulations for Wheeled Vehicles, Equipment and Parts which can be Fitted and/or be Used on Wheeled Vehicles and the Conditions for Reciprocal Recognition of Approvals Granted on the Basis of these United Nations Regulations - 4 March 2021"} and in particular on Table A1 page 20 (see Appendix A).

\section{Actors}
\noindent Three different types of actor have been identified:

\subsection*{Insider}
An insider is defined as someone who belongs to the company or works for it. They already have, therefore, access to some facilities (depending on their role in the company) and the equipment they contain (PCs, laptops, routers, etc.), the knowledge necessary to perform some tasks (internal network passwords, website and/or database credentials, VPN access, etc.) and possibly the trust of the rest of the staff.\\
They might also be aware of potential flaws in the company's systems (vulnerabilities in information system, defective equipment, weaknesses of some employees, etc.).\\
On the other hand, an insider does not necessarily have the skills needed to perform some of the intermediate attack steps needed to reach the final goal.

\subsection*{Physical intruder}
A physical intruder is different from an insider since they have less knowledge about the internals of the company. However, they have advanced \textit{physical} penetration testing skills and tools, as well as the knowledge about the company facilities location, security systems and possibly the timetables of employees and/or security personnel.\\
Therefore they might have access to a bigger number of facilities and spaces (e.g. they might have -- or be able to gain -- direct access to a server room).

\subsection*{Hacker}
A hacker is an external actor with no prior knowledge of the company's systems and no access to its facilities.
On the other hand, they have advanced security and penetration skills, in-depth knowledge of the most important and widespread technologies and attacks, as well as the tools needed to perform these attacks on the target systems.

\begin{table}[H]
	\centering
	\begin{tabular}{|c|c|c|c|}
		\hline
		\textbf{Actor}    & \textbf{Skills}                                                                                                                & \textbf{Knowledges}                                                                                                                                         & \textbf{Accesses}                                                                                            \\ \hline
		Insider           & \begin{tabular}[c]{@{}c@{}}Basic/medium IT\\ skills, social\\ engineering\end{tabular}                                         & \begin{tabular}[c]{@{}c@{}}Login credentials\\ (website, LAN,\\ VPN, DB),\\ employees info,\\ vulnerabilities in\\ company's security\\ system\end{tabular} & \begin{tabular}[c]{@{}c@{}}LAN, VPN, DB,\\ facilities, PCs,\\ workstations,\\ routers, switches\end{tabular} \\ \hline
		Physical intruder & Lockpicking, ???                                                                                                               & \begin{tabular}[c]{@{}c@{}}Facilities location,\\ personnel timetables\end{tabular}                                                                         & \begin{tabular}[c]{@{}c@{}}Company's ex-\\ ternal premises\end{tabular}                                      \\ \hline
		Hacker            & \begin{tabular}[c]{@{}c@{}}Reverse engineering,\\ social engineering,\\ advanced attack and\\ penetration testing\end{tabular} & -                                                                                                                                                           & \begin{tabular}[c]{@{}c@{}}Vehicle\\ (i.e. firmware)\end{tabular}                                            \\ \hline
	\end{tabular}
	\caption{Attackers' skills, knowledges and accesses}
	\label{tab:profiles}
\end{table}

\section{Attacker's Profile}
\noindent As we have seen each actor has its own set of knowledges, accesses and skills in table \ref{tab:profiles} it is possible to see the arrangement of them among the attackers.


\noindent For what concern the attention of the actors about the probability of detection, the cost of the attacks and the expected payoff, the weights for each of these factors are shown in the table \ref{tab:weights}.

\begin{table}[htpb]
	\centering
	\begin{tabular}{|c|c|c|c|}
		\hline
		\textbf{Attacker Name} & \textbf{Cost Weight} & \textbf{Detection Weight} & \textbf{Payoff Weight} \\ \hline
		Hacker                 & 0.1                    & 0.1                       & 0.8                    \\ \hline
		Physical Intruder      & 0.2                  & 0.3                       & 0.5                    \\ \hline
		Insider                & 0.1                  & 0.4                       & 0.5                    \\ \hline
	\end{tabular}
	\caption{Weights of Attacker Profile}
	\label{tab:weights}
\end{table}

\noindent  Thus we can see that the choices that attackers done are guided by different things, for instance an insider will prefer to avoid an attack with an high probability of detection, as well as a physical intruder, instead an hacker has not this preoccupation and he will choice the attack with the higher payoff.

\section{Goals}

\noindent In the following we will describe the goals that an attacker can achieve.

\subsection*{Vehicle Undesidered Behaviour}
The back-end servers can be used as a means to attack a vehicle and/or extract data from it e.g. its position, the destination the driver is heading to etc.

\noindent Moreover having accesses to the back-end server allow the attacker to cause an undesidered behaviour of the vehicle.

\noindent This is the most rewarding attack (reward equal to 300 in the ADVISE model) because allow the attacker to completely control the vehicles.

\subsection*{Data breach}
Servers can be also attacked to extract sensitive data related to customers.

\noindent This is the second most rewarding attack (reward equal to 150 in the ADVISE model) because the data can be very useful and valuable to the attacker, but he has no control over the vehicles.

\subsection*{Back-end server service disruption (DoS)}
An attacker could also target the back-end server just to take them down and disrupt their service (Denial of Service), causing issues to all the vehicles whose proper functioning relies on it.

\noindent It is the the attack with the lower reward value (150) because it allows the attacker only to stop the service.

\section{Countermeasures}
\noindent We have considered the following countermeasure in order to cope with the attacks that the attackers wants to perform.

\subsection*{Intrusion Detection System}
\noindent It is a software that inspects the network in order to detect unauthorized intrusion or unauthorized changes in the level of privileges, it affects the attack step \textit{Privilege Escalation} and so the Insider and the Physical Intruder. We consider two level of sensitivity: 0 (disabled) and 1 (enabled).


\subsection*{Code Obfuscation}
\noindent It is the technique that consists in the deliberate act of creating source code that is difficult for humans to understand. It affects the \textit{Firmware Reversing} attack and so the Hacker. We consider two level of sensitivity: 0 (disabled) and 1 (enabled).

\subsection*{Firewall}
\noindent We consider different qualities of firewall, it affect the \textit{Port Scan} and so the Hacker. We consider three level of sensitivity: 0 (no firewall), 1 (bad quality firewall), 2 (very good firewall).



